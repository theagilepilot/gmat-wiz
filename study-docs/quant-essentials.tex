\documentclass[11pt]{article}
\usepackage[utf8]{inputenc}
\usepackage{amsmath,amssymb}
% Map special brackets and symbols used in citations to ASCII equivalents
\DeclareUnicodeCharacter{3010}{[} % maps . to ]
\DeclareUnicodeCharacter{2020}{\dagger} % maps † to \dagger
\begin{document}
% GMAT Focus Equation Guide – reformatted into LaTeX

% This document rewrites the content of Target Test Prep's “GMAT Focus Equation Guide”
% into plain text annotated with LaTeX commands.  All of the core formulas,
% definitions and examples have been reproduced carefully and checked against
% the original PDF to ensure accuracy.  Headings are presented using
% section and subsection commands, and mathematical expressions are written
% using standard LaTeX syntax such as \frac{}, \sqrt{} and superscripts.

\section*{GMAT Focus Equation Guide}

\subsection*{Essential GMAT Focus Quant Skills}

\paragraph{Adding and subtracting fractions.}

\begin{itemize}
\item \textbf{Same denominator:} when two fractions share the same
  denominator, add or subtract their numerators and keep the denominator
  unchanged:
  \[\frac{a}{b} + \frac{c}{b} = \frac{a + c}{b}, \qquad \frac{a}{b} - \frac{c}{b} = \frac{a - c}{b}.\]
\item \textbf{Different denominators:} when denominators differ, cross‑multiply
  the numerators and denominators to produce equivalent fractions.  For
  addition
  \[\frac{a}{b} + \frac{c}{d} = \frac{ad + bc}{bd},\]
  and for subtraction
  \[\frac{a}{b} - \frac{c}{d} = \frac{ad - bc}{bd}.\]
\item \textbf{Example:} \(\tfrac{2}{3} + \tfrac{5}{7}\) has different denominators.
  Multiply the numerator and denominator of \(\frac{2}{3}\) by 7 and the
  numerator and denominator of \(\frac{5}{7}\) by 3:
  \[\frac{2}{3} + \frac{5}{7} = \frac{2\cdot7 + 5\cdot3}{3\cdot7} = \frac{14 + 15}{21} = \frac{29}{21}.\]
  Similarly,
  \(\tfrac{2}{3} - \tfrac{5}{7} = \tfrac{14 - 15}{21} = -\tfrac{1}{21}\).
\end{itemize}

\paragraph{The distributive property.}

Factoring a common factor from a sum or difference follows the distributive
property.  For variables \(a,b,c\) and constant \(d\),
\[ab + ac = a(b + c)\]..  For example,
\(4x + 4y = 4(x+y)\); the factor 4 appears in both terms and can be
extracted..

\paragraph{Multiplying and dividing fractions.}

\begin{itemize}
\item \textbf{Multiplying fractions:} multiply numerators and denominators
  separately:
  \[\frac{a}{b} \times \frac{c}{d} = \frac{ac}{bd}.\]
  \textit{Example:} \(\frac{2}{3} \times \frac{5}{7} = \frac{10}{21}.\)
\item \textbf{Dividing fractions:} multiply the dividend by the reciprocal
  of the divisor:
  \[\frac{a}{b} \div \frac{c}{d} = \frac{a}{b} \times \frac{d}{c} = \frac{ad}{bc}.\]
  \textit{Example:} \(\frac{2}{3} \div \frac{5}{7} = \frac{2}{3} \times \frac{7}{5} = \frac{14}{15}.\)
\end{itemize}

\paragraph{Reciprocals.}  The reciprocal of a non‑zero number \(x\) is
\(\frac{1}{x}\).  Multiplying a number by its reciprocal yields 1.

\paragraph{Comparing fractions: the bow‑tie method.}

To compare \(\tfrac{a}{b}\) and \(\tfrac{c}{d}\), cross‑multiply and compare
the results.  If \(ad > bc\), then \(\tfrac{a}{b} > \tfrac{c}{d}\); if
\(ad < bc\), then \(\tfrac{a}{b} < \tfrac{c}{d}\).  For example,
\(\tfrac{3}{4} > \tfrac{5}{7}\) because \(3\cdot7=21\) is greater than
\(5\cdot4=20\).

\paragraph{Converting a fraction to a percent.}

Convert the fraction to its decimal equivalent, multiply the decimal by 100
and append the percent sign..  For example,
\(\tfrac{1}{20} = 0.05\)., which is 5\,\%.

\subsection*{Linear and Quadratic Equations}

\paragraph{Factoring out common factors.}

If all terms on one side of an equation share a factor, that factor may be
factored out using the distributive property.  Examples:
\[ab + ac = a(b + c), \quad 4x + 4y = 4(x + y)..\]

\paragraph{Powers and roots of fractions.}

\begin{itemize}
\item \textbf{Square of a fraction:} \(\bigl(\tfrac{a}{b}\bigr)^2 = \tfrac{a^2}{b^2}\).
\item \textbf{Square root of a fraction:} \(\sqrt{\tfrac{a}{b}} = \tfrac{\sqrt{a}}{\sqrt{b}}\) when
  \(a,b > 0\).
\item \textbf{Number between 0 and 1:} if \(0 < x < 1\), then \(x^2 < x < \sqrt{x}\).
\end{itemize}

\paragraph{The zero‑product property.}

If the product of two factors is zero, at least one of the factors must
equal zero:
\[a\cdot b = 0 \quad\Longrightarrow\quad a = 0 \text{ or } b = 0\]
..  For instance,
\(x(x+100) = 0\) implies that either \(x=0\) or \(x+100=0\), so
\(x=0\) or \(x=-100\).

\paragraph{Quadratic equations.}

\begin{itemize}
\item The general form of a quadratic equation is
  \[ax^2 + bx + c = 0\]..
  A quadratic must be written in this form before factoring.
\item \textbf{Factoring quadratics:} for equations of the form
  \(x^2 + bx + c = 0\), find two numbers \(p\) and \(q\) such that
  \(p + q = b\) and \(pq = c\).  Then
  \[(x + p)(x + q) = 0\]..  Example:
  \[x^2 - 3x - 28 = 0 \quad\Longrightarrow\quad (x - 7)(x + 4) = 0,\]
  since \(-7\cdot 4 = -28\) and \((-7) + 4 = -3\).
\item \textbf{FOIL method:} to expand \((x - 7)(x + 4)\), multiply first,
  outside, inside, and last terms: \(x\cdot x = x^2\); \(x\cdot 4 = 4x\);
  \((-7)\cdot x = -7x\); \((-7)\cdot 4 = -28\).  Summing yields
  \(x^2 - 3x - 28\).
\item \textbf{Quadratic identities:} three common identities are
  \begin{align*}
    (x + y)^2 &= x^2 + 2xy + y^2,\\
    (x - y)^2 &= x^2 - 2xy + y^2,\\
    (x + y)(x - y) &= x^2 - y^2
  \end{align*}
  ..
\item \textbf{Difference of squares:} \(a^2 - b^2 = (a - b)(a + b)\).
  Examples include:
  \begin{align*}
    x^2 - 9 &= (x - 3)(x + 3),\\
    4x^2 - 100 &= (2x - 10)(2x + 10),\\
    x^2y^2 - 16 &= (xy - 4)(xy + 4),\\
    3^{30} - 2^{30} &= (3^{15})^2 - (2^{15})^2 \\&= (3^{15} + 2^{15})(3^{15} - 2^{15}).
  \end{align*}
\end{itemize}

\paragraph{Order of operations (PEMDAS).}

The acronym PEMDAS stands for Parentheses, Exponents, Multiplication and
Division, and Addition and Subtraction..  When
evaluating an expression, perform operations in this order.

\subsection*{Properties of Numbers}

\paragraph{Even/odd rules for addition and subtraction.}

\begin{itemize}
\item (odd) + (odd) = even..
\item (even) + (even) = even..
\item (even) + (odd) = odd..
\item (odd) - (odd) = even..
\item (even) - (even) = even..
\item (even) - (odd) = odd; (odd) - (even) = odd.
\end{itemize}

\paragraph{Multiplication rules for even and odd numbers.}

\begin{itemize}
\item even × even = even..
\item even × odd = even..
\item odd × even = even..
\item odd × odd = odd..
\end{itemize}

\paragraph{Division rules for even and odd numbers.}

In general the quotient must be an integer to be classified as even or odd.
An even integer divided by an odd integer is even; an odd divided by an odd
integer is odd; an even divided by an even integer may be even or odd
depending on how many factors of 2 remain.  An odd integer divided by an
even integer is not an integer and thus is neither even nor odd.

\paragraph{Prime numbers less than 100.}

The prime numbers under 100 are
\(2,3,5,7,11,13,17,19,23,29,31,37,41,43,47,53,59,61,67,71,73,79,83,89,97\)
..

\paragraph{Sign rules for multiplication and division.}

\begin{itemize}
\item Numbers with the same sign (both positive or both negative) yield a
  positive product or quotient: \((+)\times(+)=(+),\;(-)\times(-)=(+)\)
  ..
\item Numbers with different signs yield a negative product or quotient:
  \((+)\times(-)=(-)\)..
\end{itemize}

\paragraph{Factors and multiples.}

A factor of an integer \(x\) is an integer \(y\) that divides \(x\) evenly.
Example: the factors of 16 are 1, 2, 4, 8 and 16..
A multiple of an integer is obtained by multiplying that integer by any
integer; e.g. multiples of 4 are 4, 8, 12, 16, 20, …..

\paragraph{The division algorithm.}

For integers \(x\) and positive integer divisor \(d\), there exist unique
integers \(q\) (quotient) and \(r\) (remainder) with \(0 \le r < d\) such that
\[x = d\,q + r.\]  The possible remainders are the integers from 0
up to \(d-1\).

\paragraph{Divisibility rules.}

\begin{itemize}
\item A number is divisible by 2 if its last digit is 0, 2, 4, 6 or 8..
\item A number is divisible by 3 if the sum of its digits is divisible by 3..
\item A number is divisible by 4 if its last two digits form a number divisible by 4..
\item A number is divisible by 5 if its last digit is 0 or 5..
\item A number is divisible by 6 if it is divisible by both 2 and 3..
\item A number is divisible by 8 if its last three digits form a number divisible by 8..
\item A number is divisible by 9 if the sum of its digits is divisible by 9..
\item A number is divisible by 11 if the alternating sum of its digits (sum of digits in odd positions minus sum of digits in even positions) is divisible by 11..
\end{itemize}

\paragraph{Number of factors of an integer.}

Suppose the prime factorization of an integer is
\(p_1^{e_1} p_2^{e_2}\cdots p_k^{e_k}\).  The total number of positive
factors of the integer is
\((e_1 + 1)(e_2 + 1)\cdots(e_k + 1)\)..
Example: 240 has prime factorization
\(2^4 \cdot 3^1 \cdot 5^1\); therefore it has \((4+1)(1+1)(1+1)=20\) factors
..

\paragraph{Least Common Multiple (LCM).}

To find the LCM of integers, prime‑factorize each integer, select the
highest exponent of each prime across the factorizations, and multiply
those prime powers together..  For example, to find
\(\mathrm{LCM}(24,60)\), write
\(24 = 2^3\cdot3^1\) and \(60 = 2^2\cdot3^1\cdot5^1\).  The highest exponents
are \(2^3\), \(3^1\) and \(5^1\), so \(\mathrm{LCM} = 8\times3\times5=120\)
..

\paragraph{Greatest Common Factor (GCF).}

Prime‑factorize each integer, select the smallest exponent of each prime
common to all factorizations, and multiply those prime powers together
..  For example, \(\mathrm{GCF}(24,60)\) uses the
common primes \(2^2\) and \(3^1\); thus the GCF is \(4\times3=12\)
..  The relationship between LCM and GCF for two
positive integers \(x\) and \(y\) is
\[x\cdot y = \mathrm{LCM}(x,y)\times \mathrm{GCF}(x,y)\]..

\paragraph{Factorials and trailing zeros.}

Any factorial greater than or equal to \(5!\) ends in zero..  The number of trailing
zeros in a factorial corresponds to the number of pairs of prime factors
\(2\) and \(5\) in its prime factorization..  For example,
\(5{,}200 = 52 \times 100 = 5^2 \times (5\cdot2)^2\) has two trailing zeros
..

\paragraph{Leading zeros in reciprocals.}

If an integer \(x\) has \(k\) digits, then \(\frac{1}{x}\) will have \(k-1\)
leading zeros in its decimal representation, unless \(x\) is a power of
10, in which case there will be \(k-2\) leading zeros..

\paragraph{Terminating decimals.}

The decimal expansion of a fraction terminates precisely when, after
reducing the fraction, the denominator has no prime factors other than
2 and 5..  For instance,
\(\tfrac{1}{20} = 0.05\) terminates. whereas
\(\tfrac{1}{12} = 0.083333…\) repeats indefinitely.

\paragraph{Patterns in units digits.}

The last digit of powers of integers follows predictable cycles.:
\begin{itemize}
\item Powers of 0 always end in 0.
\item Powers of 2 repeat the cycle 2–4–8–6 (e.g. \(2^1=2,2^2=4,2^3=8,2^4=16\)).
\item Powers of 3 repeat the cycle 3–9–7–1.
\item Powers of 4 repeat the cycle 4–6: odd powers end in 4, even powers end in 6.
\item Powers of 5 always end in 5.
\item Powers of 6 always end in 6.
\item Powers of 7 repeat the cycle 7–9–3–1.
\item Powers of 8 repeat the cycle 8–4–2–6.
\item Powers of 9 repeat the cycle 9–1: odd powers end in 9 and even powers end in 1.
\end{itemize}

\paragraph{Perfect squares and cubes.}

\begin{itemize}
\item A non‑zero integer is a perfect square when all prime exponents in its
  factorization are even..  Example:
  \(144 = 2^4\times3^2\).
\item An integer is a perfect cube when all prime exponents are multiples
  of 3..  Example: \(27 = 3^3\).
\item Consecutive integers share no common prime factors, so their GCF is 1..
\end{itemize}

\paragraph{Memorizing common squares and cubes.}

Squares to memorize: \(0,1,4,9,16,25,36,49,64,81,100,121,144,169,196,225\)
..  Cubes to memorize: \(0,1,8,27,64,125,216,343,512,729,1000\)
..

\paragraph{Approximating non‑perfect square roots.}

Useful approximations include \(\sqrt{2}\approx1.4\), \(\sqrt{3}\approx1.7\)
and \(\sqrt{5}\approx2.2\).  These values can expedite rough
calculations.

\paragraph{Radicals and exponents.}

\begin{itemize}
\item \textbf{Multiplying radicals:} \(\sqrt{a}\,\sqrt{b} = \sqrt{ab}\).  For
  example, \(\sqrt{5}\,\sqrt{7} = \sqrt{35}\).
\item \textbf{Dividing radicals:} \(\frac{\sqrt{a}}{\sqrt{b}} = \sqrt{\frac{a}{b}}\).
\item \textbf{Combining like radicals:} add or subtract radicals only when
  the radicands match, e.g. \(2\sqrt{3} + 3\sqrt{3} = 5\sqrt{3}\).
\item \textbf{Square root of a square:} \(\sqrt{x^2} = |x|\).  If
  \(x\) is negative, the result becomes positive.
\item \textbf{Multiple square roots:} repeated square roots correspond to
  fractional exponents, e.g. \(\sqrt{\sqrt{x}} = x^{1/4}\).
\end{itemize}

\paragraph{Exponents to memorize.}

\begin{itemize}
\item \textbf{Base 2:} \(2^0=1, 2^1=2, 2^2=4, 2^3=8, 2^4=16, 2^5=32,
  2^6=64, 2^7=128, 2^8=256, 2^9=512, 2^{10}=1024\)
..
\item \textbf{Base 3:} \(3^1=3, 3^2=9, 3^3=27, 3^4=81, 3^5=243\)
..
\item \textbf{Base 4:} \(4^1=4,4^2=16,4^3=64,4^4=256\)
..
\item \textbf{Base 5:} \(5^1=5,5^2=25,5^3=125,5^4=625\)
..
\end{itemize}

\paragraph{Exponent rules.}

\begin{itemize}
\item \textbf{Like bases:} multiplying yields exponents that add, and dividing
  yields exponents that subtract:
  \[x^a\cdot x^b = x^{a+b}, \quad \frac{x^a}{x^b} = x^{a-b}.\]
  This rule holds for any non‑zero base \(x\)..
\item \textbf{Power to a power:} \((x^a)^b = x^{ab}\)..
\item \textbf{Different bases, same exponent:} \(x^a y^a = (xy)^a\) and
  \(\frac{x^a}{y^a} = \left(\frac{x}{y}\right)^a\)
  ..
\item \textbf{Radicals as exponents:} \(\sqrt[n]{x^m} = x^{m/n}\).  Thus
  \(\sqrt{x} = x^{1/2}\) and \(\sqrt[3]{x^2} = x^{2/3}\).
\item \textbf{Negative exponents:} for non‑zero base \(x\) and positive
  integer \(n\), \(x^{-n} = \frac{1}{x^n}\).  More generally,
  \(\left(\frac{a}{b}\right)^{-n} = \left(\frac{b}{a}\right)^n\).
\item \textbf{Special addition rules:} when expressions have the same base
  and exponent, factor them.  For example,
  \(2^{10} + 2^{11} + 2^{12} = 2^{10}(1 + 2 + 4) = 2^{10}\cdot7\)
  ..  More generally,
  \(2^n + 2^n = 2^{n+1}\), \(3^n + 3^n + 3^n = 3^{n+1}\), and
  \(4^n + 4^n + 4^n + 4^n = 4^{n+1}\)..
\end{itemize}

\paragraph{Number properties of exponents.}

Depending on the base and exponent, raising a number to a power can make
the result larger or smaller..  Important cases include:
\begin{itemize}
\item Base greater than 1 with an even or odd positive exponent >1 gives
  a result larger than the base (e.g. \(5^2>5\), \(5^3>5\)).
\item Base less than \(-1\) with an even exponent produces a larger
  (positive) result, whereas an odd exponent greater than 1 gives a
  smaller (more negative) result, e.g. \((-5)^2 > -5\) but
  \((-5)^3 < -5\)..
\item Base is a proper fraction between 0 and 1: even exponents produce a
  smaller result than the base, and odd positive exponents >1 also produce a
  smaller result.  When the base is a negative proper fraction, an even
  exponent yields a positive result which is larger than the base in
  absolute value; an odd exponent yields a negative result still larger in
  magnitude..
\item Base greater than 1 with a positive fractional exponent yields a
  smaller result than the base (e.g. \(\sqrt{5} < 5\)).  Base a proper
  fraction with a positive fractional exponent yields a larger result than
  the base (e.g. \(0.25^{1/2}=0.5>0.25\))..
\end{itemize}

\subsection*{Inequalities and Absolute Values}

\paragraph{Absolute value definition.}

The absolute value of a real number \(a\) is
\[|a| = \begin{cases}
  a, & \text{if } a \ge 0,\\
  -a, & \text{if } a < 0.
\end{cases}\]
Examples: \(|50|=50\) and \(|-50|=-(-50)=50\)..

\paragraph{Solving equations with absolute values.}

An equation of the form \(|f(x)| = c\) with \(c\ge0\) splits into two
cases: \(f(x) = c\) and \(f(x) = -c\).  Solve each separately and
combine the solutions.  When two absolute values are equal, e.g.
\(|A|=|B|\), it means either \(A = B\) or \(A = -B\).

\paragraph{Adding and subtracting absolute values.}

The triangle inequality states that \(|a| + |b| \ge |a+b|\).  Equality
occurs only when \(a\) and \(b\) have the same sign or one of them is
zero..  Similarly, \(|a| - |b| \le |a-b|\), and
equality occurs only when \(b=0\) or both \(a\) and \(b\) have the same
sign and \(|a|\ge|b|\).

\subsection*{Word Problems, Ratios and Percents}

\paragraph{Basic algebra translations.}

Common English phrases translate to mathematical operations as follows:
\begin{itemize}
\item “is,” “was,” “has been” \(\rightarrow\) equality (\(=\)).
\item “more,” “years older,” “plus” \(\rightarrow\) addition (\(+\)).
\item “less,” “years younger,” “less than,” “fewer” \(\rightarrow\) subtraction (\(-\)).
\item “times,” “of,” “factor” \(\rightarrow\) multiplication (\(\times\)).
\end{itemize}

\paragraph{Profit, cost and interest.}

\begin{itemize}
\item \textbf{Profit equation:} \(\text{Profit} = \text{Total revenue} - \text{Total cost}\).  Total cost may
  be decomposed into fixed and variable costs..
\item \textbf{Price per item:} \(\text{price} = \frac{\text{total cost}}{\text{number of items}}\).
\item \textbf{Simple interest:} \(\text{Simple Interest} = \text{Principal}\times\text{Rate}\times\text{Time}\)
 ..
\end{itemize}

\paragraph{Ratios.}

A ratio compares quantities and may be written in several equivalent ways:
\(a:b\), “\(a\) to \(b\)” or \(\tfrac{a}{b}\).  Ratios can represent part‑to‑part or
part‑to‑whole comparisons.  To combine ratios that share a common term,
scale each ratio so that the shared term takes the same numerical value,
then merge the ratios.  For example, if \(x:y = 3:4\) and \(x:z = 7:11\),
the least common multiple of the \(x\)-values (3 and 7) is 21.  Multiply the
first ratio by 7 and the second by 3 to obtain
\(x:y = 21:28\) and \(x:z = 21:33\).  Therefore \(x:y:z = 21:28:33\).

\paragraph{Percent conversions and translations.}

\begin{itemize}
\item \textbf{Converting to a percent:} multiply a decimal or integer by 100
  and append the percent sign..  For example,
  0.05 becomes 5\,\%..
\item \textbf{Converting from a percent:} remove the percent sign and divide
  by 100..  For instance, 5\,\% = 0.05.
\item \textbf{Percent of:} “\(n\) percent of \(x\)” means \(\tfrac{n}{100}\times x\).
  Thus 5\,\% of \(z\) is \(\tfrac{5}{100}z\).
\item \textbf{Percent less than:} “\(x\) is \(n\)\,\% less than \(y\)” means
  \(x = (1 - \tfrac{n}{100})y\).  Examples include 2\,\% less than
  \(y\): \(x = 0.98y\), and 60\,\% less than \(y\): \(x = 0.4y\)
 ..
\item \textbf{Percent greater than:} “\(x\) is \(n\)\,\% greater than \(y\)” means
  \(x = (1 + \tfrac{n}{100})y\).  Examples: 2\,\% greater than
  \(y\) gives \(x = 1.02y\), and 60\,\% greater than \(y\) gives
  \(x = 1.6y\)..
\item \textbf{Variable percent translations:} if \(x\) is \(n\,\%\) of \(y\), then
  \(x = \tfrac{n}{100}y\).  Likewise, if \(x\) is \(n\,\%\) less than
  \(y\), then \(x = (1 - \tfrac{n}{100})y\); if \(x\) is \(n\,\%
  \) greater than \(y\), then \(x = (1 + \tfrac{n}{100})y\).
\item \textbf{Percent change:} given an original value and a new value, the
  percent change is \(\frac{\text{new} - \text{original}}{\text{original}}\times100\,\%\).
\end{itemize}

\subsection*{Statistics and Counting}

\paragraph{Average (arithmetic mean).}

The average of a set of numbers is their sum divided by the count of the
numbers..  For evenly spaced sets, the mean equals the
average of the first and last terms.  For example, the mean of
\(\{4,5,6,7,8,9,10,11,12\}\) is \((4+12)/2 = 8\)..

\paragraph{Counting integers in consecutive sets.}

The number of integers from \(a\) to \(b\) inclusive is \(b - a + 1\)
..  To count multiples of \(k\) between \(a\) and \(b\)
inclusive, compute
\(\lfloor b/k \rfloor - \lfloor (a-1)/k \rfloor\).

\paragraph{Weighted average.}

Given values \(v_1, v_2, …, v_n\) with weights \(w_1, w_2, …, w_n\), the
weighted average is \(\frac{\sum w_i v_i}{\sum w_i}\).  The result is
closer to the value with the larger weight; a weighted average always
lies between the smallest and largest values..

\paragraph{Median and mode.}

When a set is ordered from smallest to largest, the median is the middle
value if there is an odd number of values or the average of the two middle
values if there is an even number.  The mode is the value that occurs
most often.  The range is the difference between the maximum and minimum.

\subsection*{Sequences and Series}

\paragraph{Arithmetic sequences.}

A sequence is arithmetic if each term after the first differs from the
previous term by a constant difference \(d\).  The \(n\)‑th term of an
arithmetic sequence is
\[a_n = a_1 + (n-1)d,\]
and the sum of the first \(n\) terms is
\[S_n = \frac{n}{2}\bigl(2a_1 + (n-1)d\bigr) = \frac{n}{2}(a_1 + a_n).\]

\paragraph{Geometric sequences.}

A sequence is geometric if successive terms are obtained by multiplying
the previous term by a constant ratio \(r\).  The \(n\)-th term is
\(a_n = a_1 r^{n-1}\).  The sum of the first \(n\) terms (\(r\ne1\)) is
\[S_n = a_1\,\frac{1 - r^n}{1 - r}.\]
If \(|r| < 1\), the sum to infinity is \(S_\infty = \frac{a_1}{1 - r}\).

\paragraph{Geometric progression example.}

In the geometric progression 5, 10, 20, 40, … each term is obtained by
multiplying the preceding term by 2.  The common ratio is 2 and the sum of
the first \(n\) terms is \(5\,(2^n - 1)\).

\subsection*{Probability and Combinatorics}

\paragraph{Factorials and circular arrangements.}

\begin{itemize}
\item The factorial of a non‑negative integer \(n\) is defined as
  \(n! = n(n-1)(n-2)\cdots1\), with \(0! = 1\).
\item The number of ways to arrange \(k\) distinct objects in a circle is
  \((k-1)!\)..
\end{itemize}

\paragraph{Permutations and combinations.}

\begin{itemize}
\item The number of permutations of \(k\) objects chosen from \(n\) distinct
  objects is
  \[P(n,k) = \frac{n!}{(n-k)!}.\]
\item The number of combinations of \(k\) objects chosen from \(n\) distinct
  objects (order does not matter) is
  \[C(n,k) = \binom{n}{k} = \frac{n!}{k!(n-k)!}.\]
\end{itemize}

\paragraph{Basic probability.}

The probability of an event \(A\) is the ratio of the number of favorable
outcomes to the total number of equally likely outcomes.  The sum of
probabilities across the entire sample space equals 1..  If
\(P(A)\) denotes the probability of \(A\), then the probability of the
complement (“not \(A\)”) is \(1 - P(A)\)..

\paragraph{Multiplication and addition rules.}

\begin{itemize}
\item For independent events \(A\) and \(B\), \(P(A\text{ and }B) = P(A)P(B)\)
 ..  For events that are not independent,
  \(P(A\text{ and }B) = P(A)P(B\mid A)\)..
\item For mutually exclusive events, \(P(A\text{ or }B) = P(A) + P(B)\)
 ..  If events are not mutually exclusive,
  use
  \begin{equation*}P(A\text{ or }B) = P(A) + P(B) - P(A\text{ and }B).\end{equation*}
\item The probability that at least one of several independent events
  occurs is one minus the probability that none of them occur:
  \begin{equation*}P(\text{at least one}) = 1 - P(\text{none}).\end{equation*}.
\end{itemize}

\subsection*{Closing Remarks}

This LaTeX document consolidates the formulas, definitions and examples from
the Target Test Prep “GMAT Focus Equation Guide” into a single, searchable
plain‑text resource.  Wherever possible, formulas have been restated in
canonical algebraic form and accompanied by illustrative examples.  Each
statement has been checked against the original PDF to ensure fidelity to
the source material.
\end{document}

